\documentclass{article}
\usepackage{amsmath}
\usepackage[utf8]{inputenc}
\DeclareMathOperator{\sign}{sign}
\begin{document}

\section{Normal approximation}

The underlying process is constructed as a sequence of~$N$ Bernoulli trials with success rate~$\tilde{p}$. Under the normal approximation, the PDF of the observed success rate~$f(p)$ is given by:
\begin{equation}
f(p) \sim N \left( \tilde{p}, \,\,\, \sqrt[]{\frac{\tilde{p} (1 - \tilde{p})}{N}} \right)
\end{equation}

\section{One proportion test}

Let's assume that we prepared an actual realization of~$N$~Bernoulli trials.  Our objective is to assign a p-value telling us the probability that the success rate of the observed realization is in fact higher than some test rate~$p_0$. As we saw above (TODO) the first step is to calculate the z-score according to:
\begin{equation}
z = \frac{p - p_0}{\sqrt[]{p_0 (1 - p_0)}} \, \sqrt[]{N}
\end{equation}

\noindent where~$p_0$ is the rate that we are testing against.  A small calculation show that the PDF of the observed z-scores should behave (under the normal approximation) as:
\begin{equation}
f(z) \sim N \left( \mu_z , \,\,\, \sigma_z \right) \quad \mbox{with} \quad 
\begin{cases}
    \mu_z  & = \left( \tilde{p}  - p_0 \right) \sqrt[]{\frac{N}{p_0 (1 - p_0)}} \\
    \sigma_z & = \sqrt[]{\frac{\tilde{p} (1 - \tilde{p})}{p_0 (1 - p_0) }}
  \end{cases}
\end{equation}

\section{One tailed p-value}

One can then extract the one tailed p-value from the z-score as follows:
\begin{equation}
p = 1 - \Phi \left( z \right)
\end{equation}

\noindent where $\Phi$ stands for the cumulative distribution function of the standard normal distribution~$N(0, 1)$.  Since there are no closed forms for~$\Phi$ and its inverse the quantile function, one has to rely on error functions (and their inverses).  Instead, We can simplify (and maybe gain more insight) by using the P\'{o}lya approximation:
\begin{equation}
\Phi(z) \approx \frac{1}{2} \left[ 1 + \sign(z) \, \, \sqrt[]{1 - \exp \left( - \frac{2 z^2}{\pi} \right)} \right]
\end{equation}

\noindent In addition to its simplicity, this closed form approximation has the advantage that it can easily be inverted such that we eventually get:
\begin{equation}
z(p) \approx \sign \left( \frac{1}{2} - p \right) \sqrt[]{\frac{\pi}{2} \log \frac{1}{4 p (1 - p) }}
\end{equation}

\noindent In the limit~$p \ll 1$, we can simplify the inverse relation to:
\begin{equation}
z(p) \approx \sqrt[]{\frac{\pi}{2} \log \frac{1}{4p}} \quad ; \quad \left| \frac{dz}{dp} \right| \approx \sqrt[]{\frac{\pi}{2}} \,\, \frac{1}{2 p \, \, \sqrt[]{\log 1 / 4p}}
\end{equation}


\end{document}

